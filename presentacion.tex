\documentclass{beamer}
\usetheme{PaloAlto}
\usepackage[spanish]{babel}
\usepackage{tcolorbox}
\usepackage[utf8]{inputenc}
\usepackage{graphicx}
\usepackage{xcolor}

\definecolor{studentbrown}{RGB}{124,71,50}
\definecolor{studentgreen}{RGB}{31,78,58}
\definecolor{studentpurple}{RGB}{41,31,78}

\newtcolorbox{mybox}[1][Theorem:]{
colback=white,
colbacktitle=white,
coltitle=red!70!black,
colframe=red!70!black,
boxrule=1pt,
titlerule=0pt,
arc=15pt,
title={\strut#1}
}

\newenvironment{variableblock}[3]{%
  \setbeamercolor{block body}{#2}
  \setbeamercolor{block title}{#3}
  \begin{block}{#1}}{\end{block}}

\title{Estudio de complejidad y aproximabilidad}
\author{Aitor Godoy Fresneda}
\institute{Universidad Complutense de Madrid \\ Facultad de Ciencias Matem\'aticas}
\date{Julio 2020}

\begin{document}

\begin{frame}
\maketitle
\end{frame}

%%%  ÍNDICE

\begin{frame}
\frametitle{\'Indice}
\tableofcontents
\end{frame}


\newpage

%%%  SECCIÓN 1

\begin{frame}
\frametitle{Motivaci\'on.}
\section{Motivaci\'on.}

\begin{itemize}
\item ¿P = NP?
\item Soluci\'ones \'optimas para problemas NP-Completos costosas.
\item Algoritmos de aproximaci\'on polin\'omicos.
\end{itemize}

\end{frame}

%%%  SECCIÓN 2

\begin{frame}{Preliminares.}
\section{Preliminares.}
\begin{block}{La clase P.}
Decimos que un problema pertenece a la clase P cuando existe una m\'aquina de Turing determinista que lo resuleve en tiempo polin\'omico.
\end{block}

\begin{variableblock}{Ejemplo.}{bg=studentgreen!20,fg=black}{bg=studentgreen,fg=white}
\begin{itemize}
\item El problema de primalidad.
\item Problema de ordenaci\'on de un vector de enteros.
\item Encontrar el camino m\'as corto entre 2 v\'ertices de un grafo.
\end{itemize}
\end{variableblock}

\end{frame}

\begin{frame}{Preliminares.}

\begin{block}{La clase NP.}
Decimos que un problema pertenece a la clase NP cuando existe una m\'aquina de Turing no determinista que lo resuelve en tiempo polin\'omico.
\end{block}

\begin{columns}[T] % align columns
\begin{column}{.46\textwidth}
\begin{variableblock}{Ejemplo.}{bg=studentgreen!20,fg=black}{bg=studentgreen,fg=white}
\begin{itemize}
\item El problema de la mochila.
\item El problema MIN TSP.
\item El problema CLIQUE.
\end{itemize}
\end{variableblock}
\end{column}%
\hfill%
\begin{column}{.50\textwidth}
\begin{variableblock}{Observaciones.}{bg=studentpurple!20,fg=black}{bg=studentpurple,fg=white}
\begin{itemize}
\item P $\subseteq$ NP.
\item La soluci\'on se puede comprobar en tiempo polin\'omico.
\end{itemize}
\end{variableblock}
\end{column}%
\end{columns}

\end{frame}

\begin{frame}{Preliminares.}

\begin{columns}[T]
\begin{column}{.46\textwidth}
\begin{block}{Reducciones polin\'omicas.}
\begin{itemize}
\item Transformaci\'on de $P_1$ a $P_2$ en tiempo polin\'omico.
\item Transforma entradas de $P_1$ a $P_2$ en tiempo polin\'omico.
\item Nueva entrada polin\'omica respecto de la entrada de $P_1$.
\end{itemize}
\end{block}
\end{column}
\begin{column}{.50\textwidth}
\centering
\includegraphics[width=0.6\textwidth,height=.70\textheight]{reduccion.png}
\end{column}
\end{columns}

\end{frame}

\begin{frame}{Preliminares.}

\begin{columns}[T]

\begin{column}{.46\textwidth}
\begin{block}{Problemas NP-Duros y NP-Completos.}
\begin{itemize}
\item Un problema P es NP-Duro si para todo problema P' $\in$ NP existe una reducci\'on polin\'omica de P' a P
\item Un problema P es NP-Completo si P es NP-Duro y P $\in$ NP
\end{itemize}
\end{block}
\end{column}

\begin{column}{.50\textwidth}
\begin{variableblock}{Ejemplo.}{bg=studentgreen!20,fg=black}{bg=studentgreen,fg=white}
\begin{itemize}
\item El problema de la mochila.
\item El Problema de PCI.
\item El problema SAT.
\end{itemize}
\end{variableblock}

\centering
\includegraphics[width=0.7\textwidth,height=.35\textheight]{PyNP.jpg}

\end{column}
\end{columns}

\end{frame}

\begin{frame}{Preliminares.}

\begin{columns}[T]

\begin{column}{.46\textwidth}
\begin{variableblock}{Ejemplo.}{bg=studentgreen!20,fg=black}{bg=studentgreen,fg=white}
\begin{itemize}
\item El Teorema de COOK.
\item Reducci\'on de SAT a 3-SAT.
\item Reducci\'on de 3-SAT a PCI.
\item Reducci\'on de PCI a CLIQUE.
\end{itemize}
\end{variableblock}
\end{column}

\begin{column}{.50\textwidth}

\centering
\includegraphics[width=0.8\textwidth,height=.75\textheight]{reducciones2.png}

\end{column}
\end{columns}
    
\end{frame}

%%%%%  SECCIÓN 3.

\begin{frame}{Clases de aproximaci\'on.}

\section{Clases de aproximaci\'on.}

\begin{itemize}
\item Contraparte de optimizaci\'on de los problemas de decisi\'on.
\item Buscamos soluciones ``buenas'' en tiempo polin\'omico.
\item Las diferentes clases tienen una jerarqu\'ia.
\end{itemize}

\begin{block}{La clase NPO.}
\begin{itemize}
\item Contraparte de optimizaci\'on de los problemas NP.
\item 4-tupla $(I, Sol, m, obj)$.
\item Distinci\'on entre maximizaci\'on y minimizaci\'on.
\end{itemize}
\end{block}
    
\end{frame}

\begin{frame}{Clases de aproximaci\'on.}
Dos paradigmas de aproximaci\'on principales:

\begin{columns}[T]

\begin{column}{.46\textwidth}
\begin{block}{Paradigma de aproximaci\'on est\'andar.}
\begin{itemize}
\item $\rho_A(I)=\frac{m_A(I,S)}{opt(I)}$.
\item $[1,\infty)$ para minimizaci\'on.
\item $(0,1]$ para maximizaci\'on.
\end{itemize}
\end{block}
\end{column}

\begin{column}{.44\textwidth}
\begin{block}{Paradigma de aproximaci\'on diferencial.}
\begin{itemize}
\item $\delta_A(I)=\frac{|\omega(I)-m_A(I,S)|}{|\omega(I)-opt(I)|}$.
\item $[0,1]$ para minimizaci\'on.
\item $[0,1]$ para maximizaci\'on.
\end{itemize}
\end{block}
\end{column}
\end{columns}

\end{frame}

\begin{frame}{Clases de aproximaci\'on.}

Clases poco optimistas.

\begin{block}{Exp-APX y Exp-DAPX.}
El coeficiente de aproximaci\'on crece (o decrece) de forma exponencial
\end{block}

\begin{block}{Poly-APX y Poly-DAPX.}
El coeficiente de aproximaci\'on crece (o decrece) de forma polin\'omica.
\end{block}

\begin{block}{Log-APX y Log-DAPX.}
El coeficiente de aproximaci\'on crece (o decrece) de forma logar\'itmica.
\end{block}

\end{frame}


\begin{frame}{Clases de aproximaci\'on.}
Clases m\'as optimistas.

\begin{block}{APX y DAPX.}
El coeficiente de aproximaci\'on es constante.
\end{block}

\begin{block}{PTAS y DPTAS.}
Podemos aproximar nuestra soluci\'on a la \'optima tanto como queramos de modo que respecto del tamaño del problema obtengamos un algoritmo en tiempo polin\'omico pero respecto de cu\'anto queremos aproximar no necesariamente.
\end{block}

\begin{block}{FPTAS y DFPTAS.}
FPTAS es como PTAS pero si que es polin\'omico respecto de cu\'anto lo quieres aproximar.
\end{block}

\end{frame}


\begin{frame}{Clases de aproximaci\'on.}

Las clases de aproximaci\'on presenta una jerarqu\'ia:

\begin{columns}[T]

\begin{column}{.50\textwidth}
\includegraphics[width=\textwidth,height=.50\textheight]{Hierar1.png}
\end{column}

\begin{column}{.46\textwidth}
\includegraphics[width=\textwidth,height=.50\textheight]{Hierar2.png}
\end{column}
\end{columns}
    
\end{frame}

%%% SECCIÓN  4.

\begin{frame}{Aproximabilidad en problemas NP-Duros.}

\section{Aproximabilidad en problemas NP-Duros.}

\begin{columns}[T]

\begin{column}{.46\textwidth}
\begin{variableblock}{El problema de la mochila.}{bg=studentgreen!20,fg=black}{bg=studentgreen,fg=white}
Est\'a en FPTAS. El algoritmo de aproximaci\'on es el siguiente:
\begin{enumerate}
\item Fijamos un $\varepsilon \in (0,1)$ y consideramos la entrada $I'=((a_i',c_i)_{i=1,...,n},b)$ con $a_i'=\lfloor\frac{a_i}{t}\rfloor$ con $t=\frac{a_{max}\varepsilon}{n}$.
\item Salida S := PROGRAMACI\'ON DIN\'AMICA(I').
\end{enumerate}
\end{variableblock}
\end{column}

\begin{column}{.50\textwidth}

\includegraphics[width=\textwidth,height=.80\textheight]{KNAPSACK2.png}

\end{column}
\end{columns}

\end{frame}


\begin{frame}{Aproximabilidad en problemas NP-Duros.}

\begin{columns}[T]

\begin{column}{.36\textwidth}
\centering
\includegraphics[width=\textwidth,height=.70\textheight]{PCI.png}
\end{column}

\begin{column}{.60\textwidth}

\begin{variableblock}{El problema del Conjunto Independiente maximal (PCI).}{bg=studentgreen!20,fg=black}{bg=studentgreen,fg=white}
Est\'a en Poly-Apx. Algoritmo IS:
\begin{enumerate}
\item Resolver la relajaci\'on lineal del PCI para hallar $V_0$, $V_1$ y $V_{1/2}$.
\item Colorear $G[V_{1/2}]$ con a lo sumo $\Delta(G[V_{1/2}])$ colores. y definir $\hat{S}$ como el subconjunto de v\'ertices de $G[V_{1/2}]$ cuyo color es el que m\'as v\'ertices cubre del grafo.
\item Devolver $S = V_1 \bigcup \hat{S}$
\end{enumerate}
\end{variableblock}

\end{column}
\end{columns}
    
\end{frame}


\begin{frame}{Aproximabilidad en problemas NP-Duros.}

\begin{columns}[T]

\begin{column}{.46\textwidth}
\begin{variableblock}{MIN VERTEX COVER (MVC).}{bg=studentgreen!20,fg=black}{bg=studentgreen,fg=white}
Est\'a en Apx. Algoritmo MATCHING:
\begin{enumerate}
\item Computar un matching maximal en G.
\item Devolver el conjunto C de los v\'ertices de las aristas de M.
\end{enumerate}
\end{variableblock}
\end{column}

\begin{column}{.50\textwidth}
\centering
\includegraphics[width=0.8\textwidth,height=.70\textheight]{MVC.png}
\end{column}
\end{columns}

\end{frame}


\begin{frame}{Aproximabilidad en problemas NP-Duros.}

\begin{columns}[T]

\begin{column}{.40\textwidth}
\includegraphics[width=0.75\textwidth,height=.35\textheight]{tsp1.png}
\includegraphics[width=0.75\textwidth,height=.45\textheight]{tsp3.png}
\end{column}

\begin{column}{.56\textwidth}
\begin{variableblock}{MIN TSP.}{bg=studentgreen!20,fg=black}{bg=studentgreen,fg=white}
Est\'a en DAPX. Algoritmo 2\_OPT:
\begin{enumerate}
\item Construir un ciclo hamiltoniano $T$.
\item consideramos dos aristas $a_{ij}$, $a_{i'j'}$ y si procede cambiarlas las cambiamos por $a_{ii'}$, $a_{jj'}$.
\item repetir el paso 2 hasta que no haya que cambiar ninguna arista.
\item Devolver $T$.
\end{enumerate}
\end{variableblock}
\end{column}
\end{columns}

\end{frame}

%%% SECCIÓN 5.

\begin{frame}{Aproximabilidad mediante reducciones.}

\section{Aproximabilidad mediante reducciones.}

\begin{itemize}
\item Reducciones polin\'omicas para problemas de optimizaci\'on.
\item Preservan aproximabilidad.
\item Transformaci\'on del coeficiente de aproximabilidad.
\end{itemize}


\end{frame}


\begin{frame}{Aproximabilidad mediante reducciones.}

\begin{block}{Reducciones que preservan aproximabilidad.}
Dados dos problemas NPO $\Pi$ y $\Pi'$ , una reducci\'on que preserva aproximabilidad R  de $\Pi$ a $\Pi'$ ($\Pi \leq_{R} \Pi'$) es una terna $(f, g, c)$ de funciones cumputables en tiempo polin\'omico tal que:

\begin{itemize}

\item $f$ transforma una entrada $I\in\textbf{I}$ en una entrada $f(I)\in\textbf{I'}$.
\item $g$ transforma una soluci\'on de $\Pi'$ en una soluci\'on de $\Pi$.
\item $c$ transforma el coeficiente de aproximaci\'on de aproximaci\'on de $\Pi'$ en uno de $\Pi$.

\end{itemize}
\end{block}
    
\end{frame}


\begin{frame}{Aproximabilidad mediante reducciones.}

Esta clase de reducciones son muy \'utiles principalmente por dos motivos:


\begin{itemize}
    \item Por motivos estructurales:
    
    - Gerarqu\'ia en los problemas NPO.
    
    - No todos los problemas son igual de dif\'iciles de aproximar.
\end{itemize}
\begin{itemize}
    \item Por motivos operacionales:
    
    - No siempre se puede hallar aproximabilidad directamente.
    
    - Herramienta muy \'util para hallar nuevos resultados.
\end{itemize}

\end{frame}


\begin{frame}{Aproximabilidad mediante reducciones.}

\begin{columns}[T]
\begin{column}{.46\textwidth}
\begin{variableblock}{Ejemplos.}{bg=studentgreen!20,fg=black}{bg=studentgreen,fg=white}
\begin{enumerate}
\item Reducci\'on de MAX PCIV a MAX PCI.
\item Reducci\'on de MIN COLORING a MIN Partition Into Independent Sets.
\item Reducci\'on de MAX PCI a MAX CLIQUE.
\end{enumerate}
\end{variableblock}
\end{column}

\begin{column}{.50\textwidth}
\centering
\includegraphics[width=0.5\textwidth,height=.35\textheight]{redux1.png}
\includegraphics[width=0.7\textwidth,height=.40\textheight]{redux2.png}
\end{column}
\end{columns}

\end{frame}

%%% SECCIÓN 6.

\begin{frame}{Conclusiones.}

\section{Conclusiones.}

\begin{itemize}
\item La aproximabilidad es muy \'util ya que no sabemos si $P=NP$.
\item Necesaria ya que no se ha conseguido encontrar algoritmos ``r\'apidos'' para algunos problemas a la hora de hallar la soluci\'on \'optima.
\item Nos ayuda a comprender el mundo que nos rodea.
\item No siempre vamos a poder hallar la mejor soluci\'on a nuestros problemas y vamos a tener que encontrar otra manera de hallar una soluci\'on lo suficientemente buena
\end{itemize}
    
\end{frame}

\end{document}
